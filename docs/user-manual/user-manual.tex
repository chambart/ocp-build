%%%%%%%%%%%%%%%%%%%%%%%%%%%%%%%%%%%%%%%%%%%%%%%%%%%%%%%%%%%%%%%%%%%%%%%%%%
%                                                                        %
%                        TypeRex OCaml Studio                            %
%                                                                        %
%                 Thomas Gazagnaire, Fabrice Le Fessant                  %
%                                                                        %
%  Copyright 2011-2012 OCamlPro                                          %
%  All rights reserved.  This file is distributed under the terms of     %
%  the GNU Public License version 3.0.                                   %
%                                                                        %
%  TypeRex is distributed in the hope that it will be useful,            %
%  but WITHOUT ANY WARRANTY; without even the implied warranty of        %
%  MERCHANTABILITY or FITNESS FOR A PARTICULAR PURPOSE.  See the         %
%  GNU General Public License for more details.                          %
%                                                                        %
%%%%%%%%%%%%%%%%%%%%%%%%%%%%%%%%%%%%%%%%%%%%%%%%%%%%%%%%%%%%%%%%%%%%%%%%%%

\chapter{Building OCaml Projects with {\tt ocp-build}}

{\tt ocp-build} can be used to compile simple OCaml projects.
The tool uses simple configuration files to describe the
packages that need to be compiled, and the dependencies between
them.

Compared to other OCaml building tools, it provides the following
particularities:
\begin{itemize}
\item {\tt ocp-build} supports complete parallel builds. Its improved
  understanding of OCaml compilation constraints avoids traditionnal
  problems, arising from conflicts while compiling interfaces.
\item {\tt ocp-build} configuration files provide a simple and concise
  way to handle the complexity of OCaml projects.
\item {\tt ocp-build} supports complex compilation rules, such as
  per-file options, packing and C stubs files.
\item {\tt ocp-build} can use either a set of attributes or a digest
  of the content of a file to detect files' modifications to decide
  which files should be rebuilt.
\end{itemize}

\section{Configuration Files}

 {\tt ocp-build} uses two different kind of files to describe a project:
\begin{itemize}
\item Each package (or set of packages) should be described in a file
  with an {\tt .ocp} extension. When {\tt ocp-build} is run with the
  {\tt -scan} option, it scans the directory to find all such
  configuration files, and adds them to the project.
\item The project should be described in a file {\tt
  ocp-build.root}. This file should be at the root of the project, and
  {\tt ocp-build} will try to find it by recursively scanning all the
  parents directories. If it does not exist, it should be created using
  the {\tt -init} option.
\end{itemize}

\section{Compilation Layout}

{\tt ocp-build} generates files both in the source directories and in
a special {\tt \_obuild} directory, depending on the nature of the
files:
\begin{itemize}
\item Temporary source files and compilation garbage are stored in the
  source directories. This set includes implementation and interfaces
  files generated by {\tt ocamllex} and {\tt ocamlyacc}, and other
  special files such as {\tt .annot} files.
\item Binary object files are stored in the {\tt \_obuild} directory,
  where a sub-directory is created for each package.
\end{itemize}

\section{Format of the package description files ({\tt .ocp})}

\subsection{Description of Simple Packages}

A simple package description looks like this:

\begin{verbatim}
begin library "ocplib-system"
  files = [ "file.ml" "process.ml" ]
  requires = [ "unix" ]
end
\end{verbatim}

This description explains to {\tt ocp-build} that a library {\tt
  ocplib-system} should be built from source files {\tt file.ml} and
{\tt process.ml} (and possibly {\tt file.mli} and {\tt process.mli}),
and that this library depends on the {\tt unix} library to be built.

Another simple description is:

\begin{verbatim}
begin program "file-checker"
  files = [ "checkFiles.ml" "checkMain.ml" ]
  requires = [ "ocplib-system" ]
end
\end{verbatim}

This description tells {\tt ocp-build} that it should build an
executable {\tt file-checker} from the provided source files, and with
a dependency towards {\tt ocplib-system}. {\tt ocp-build} will
automatically add the dependency towards {\tt unix} required by {\tt
  ocplib-system}.

\subsection{Advanced options}

\subsubsection{Per-file options}

Options can be specified on a per-file basis:

\begin{verbatim}
begin library "ocplib-fast"
  files = [
    "fastHashtbl.ml" (asmcomp = [ "-inline"; "30" ])
    "fastString.ml"
  ]
end
\end{verbatim}

They can also be specified for a group of files:

\begin{verbatim}
begin library "ocplib-fast"
  files = [
    begin  (asmcomp = [ "-inline"; "30" ])
    "fastHashtbl.ml"
    "fastMap.ml"
    end
    "fastString.ml"
  ]
end
\end{verbatim}

\subsubsection{Configurations}

\subsubsection{Preprocessor requirements: {\tt pp\_requires}}

The {\tt pp\_requires} option can be used to declare a dependency
between one or more source files and a preprocessor that should thus
be built before. The preprocessor must be specified as a program
package in a projet, plus the target (bytecode {\tt byte} or native
{\tt asm}):

\begin{verbatim}
begin library "ocplib-doc"
  files = [
    "docHtml.ml" (
       pp = [ "./_obuild/ocp-pp/ocp-pp.byte ]
       pp_requires = [ "ocp-pp:byte" ]
    )
    "docInfo.ml"
  ]
  requires = ["ocp-pp"]
end
\end{verbatim}

Note that you still need:
\begin{itemize}
\item To specify the package in the {\tt requires} directive, to ensure that
  this package will be available when your package will need it for processing.
\item To specify the command {\tt pp} to call the preprocessor
\end{itemize}

\section{Command line options}

